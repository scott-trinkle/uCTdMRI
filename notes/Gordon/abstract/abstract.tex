\documentclass[11pt]{article}
\usepackage[
top    = 1.5cm,
bottom = 1.5cm,
left   = 1.5cm,
right  = 1.5cm]{geometry}
\usepackage{setspace}
\setstretch{1.5}
\usepackage{amsmath}
\usepackage{upgreek}


\title{\vspace{-2em}Towards whole-brain validation of diffusion MRI fiber
  orientation distributions with x-ray microcomputed tomography}
\author{Scott Trinkle, Sean Foxley, Narayanan Kasthuri and Patrick La Rivi\`ere}
\date{Revised: \today}

\begin{document}
\maketitle

Diffusion MRI (dMRI) is a powerful, non-invasive tool for characterizing
three-dimensional (3D) tissue microstructure on a macroscopic scale, and is
widely used in both research and clinical settings. New methods of
reconstructing 3D fiber orientation distributions (FODs) from dMRI data are
rapidly being developed, each based on the assumption that the diffusion
contrast from dMRI provides an accurate representation of the underlying
anatomical fiber structure. Previous efforts to validate these FODs have relied
on ground-truth histological data with non-isotropic resolution over small
regions of interest (ROI). In this study, we demonstrate a pipeline for the use
of synchrotron-based x-ray microcomputed tomography data to
validate FODs from dMRI over a whole mouse brain with isotropic resolution.

A perfusion-fixed mouse brain was scanned on a Bruker 9.4 T magnet with a 3D
diffusion-weighted spin-echo sequence at 150 $\upmu$m isotropic resolution. Data
were acquired at a b-value of 3000 s/mm$^2$ over 30 uniformly distributed
directions. The specimen was then stained with uranyl acetate, osmium tetroxide,
and lead citrate in preparation for synchrotron x-ray imaging at the Advanced
Photon Source at Argonne National Lab. The x-ray data were acquired using a
mosaic sinogram stitching method, yielding a reconstructed image volume over the
whole brain with 2.4 $\upmu$m isotropic resolution.

Structure tensor analysis was performed on the x-ray data to estimate a primary
orientation vector at each voxel. These orientations were then grouped into ROI
the size of a single MRI voxel, and ground-truth FODs were computed by expanding
the distribution of orientation vectors within each ROI onto spherical harmonic
coefficients. A sensitivity study was performed with simulated phantoms containing populations
of fibers with various known crossing angles. Angular peaks in the FODs
calculated using structure tensor analysis were within two degrees of the true
peaks for all fiber crossing angles. 

This work introduces a ground truth dataset and methodology that will allow for
whole-brain validation of dMRI FODs with natively isotropic resolution and no
sectioning. Once the X-ray and dMRI data are registered, FODs can be
reconstructed from the dMRI data using a number of available
algorithms. Quantitative comparisons to the X-ray FODs across the whole brain
will provide a wealth of information regarding the ability of these algorithms
to represent microstructural regions of varying complexity, and will provide the
means to perform future large-scale validation studies in dMRI, tractography,
and connectomics.

\end{document}