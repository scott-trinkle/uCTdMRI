\documentclass[11pt]{article}
\usepackage[
top    = 1.5cm,
bottom = 1.5cm,
left   = 1.5cm,
right  = 1.5cm]{geometry}
\usepackage{setspace}
\setstretch{1.5}
\usepackage{amsmath}

\title{Towards whole-brain validation of diffusion MRI fiber orientation
  distributions with x-ray microcomputed tomography}
\author{Scott Trinkle, Sean
  Foxley, Narayanan Kasthuri and Patrick La Rivi\`ere}
\date{}

\begin{document}
\maketitle

Diffusion MRI (dMRI) is a powerful, non-invasive tool for characterizing
three-dimensional (3D) tissue microstructure on a macroscopic scale, and is
widely used in both research and clinical settings. New methods of
reconstructing 3D fiber orientation distributions (FODs) from dMRI data are
rapidly being developed, each based on the assumption that dMRI provides an
accurate model of the underlying anatomical fiber structure. Previous efforts to
validate these FODs have relied on ground truth histological data with
non-isotropic resolution over small regions of interest (ROI). In this study, we
demonstrate a pipeline for the use of natively isotropic, synchrotron-based x-ray
microcomputed tomography data to validate FODs over a whole mouse brain.

A post-mortem brain was scanned with dMRI (\textit{waiting on MRI specs from
  Sean}) at 150 $\mu\text{m}$ isotropic resolution. The specimen was then
stained (\textit{what are the stains? Other info on uCT sample preparation?})
and imaged at the Advanced Photon Source at Argonne National Lab using a mosaic
stitching method, yielding an image volume over the whole brain with 1.2 $\mu$m
isotropic resolution. N (\textit{actual number TBD}) sample ROIs were identified
for validation based on anatomical locations with crossing fiber populations
that are known to challenge dMRI algorithms. Structure tensor analysis was
performed on the x-ray data to compute a ground truth FOD at each ROI. The
corresponding FODs from the dMRI data were evaluated based on overall agreement
in FOD shape, correct assessment of the number of fiber populations, and angular
accuracy in orientation (\textit{criteria taken verbatim from Schilling}).

(\textit{Results TBD. I think the goal should be to demonstrate comparable
  results to previous validation studies over a few ROI, to justify the use of
  this dataset/method for a future whole-brain study.})

This study has demonstrated the feasibility of performing quantitative 3D
validation of dMRI FODs with synchrotron x-ray data. Applying this analysis to a
whole mouse brain will provide a wealth of information regarding the ability of
different dMRI algorithms to represent microstructural regions of varying
complexity (\textit{It might be cool to one day have a spatial map over the whole
  brain of where the MRI is failing}). Ground truth FODs across the whole brain
will allow for future large-scale validation studies in both dMRI and
tractography.

\end{document}